\documentclass{beamer}
\usepackage[utf8]{inputenc}
\usepackage[russian]{babel}
\usepackage{amsmath,mathrsfs,mathtext}
\usepackage{graphicx, epsfig}
\usetheme{Warsaw}%{Singapore}%{Warsaw}%{Warsaw}%{Darmstadt}
\usecolortheme{sidebartab}
\definecolor{beamer@blendedblue}{RGB}{15,120,80}
%----------------------------------------------------------------------------------------------------------
\title[\hbox to 56mm{ Notes to \LaTeX \hfill\insertframenumber\,/\,\inserttotalframenumber}]
{Web-сервис для конвертирования рукописных конспектов  в  \LaTeX-документ}
\author[Т. Гадаев, Г. Моргачев, Е. Шульгин]{\large \\Т. Гадаев, Г. Моргачев, Е. Шульгин}
\institute{\large
Московский физико-технический институт  \\ Факультет управления и прикладной математики}

\date{\footnotesize{\emph{Курс:} Основы объектно ориентированного программирования \par (лабораторные, А.\,И. Панов)/Группа 571, весна 2017}}
%----------------------------------------------------------------------------------------------------------
\begin{document}
%----------------------------------------------------------------------------------------------------------
\begin{frame}
%\thispagestyle{empty}
\titlepage
\end{frame}
%----------------------------------------------------------------------------------------------------------
\begin{frame}{Идея проекта}
    \begin{itemize}
        \item Web-сервис для оцифровки конспектов и последующего быстрого редактирования
        \item Распознавание текста, формул и графиков с помощью нейросетевых технологий
        \item Совместное редактирование
    \end{itemize}
\end{frame}
%-----------------------------------------------------------------------------------------------------
\begin{frame}{Распределение по задачам}
    
\begin{block}{Гадаев Тамаз - тимлид}
    Распознавание
\end{block}
\begin{block}{Моргачев Глеб}
    Back-end
\end{block}
\begin{block}{Шульгин Егор}
    Front-end
\end{block}
\end{frame}
%----------------------------------------------------------------------------------------------------------
\begin{frame}{Аналоги}
    \textbf{JMathNotes}
        \begin{figure}
            \centering
            \includegraphics[width=0.7\linewidth]{../../../Downloads/jmn-300x221}
            \label{fig:jmn-300x221}
        \end{figure}
\end{frame}
%----------------------------------------------------------------------------------------------------------
\begin{frame}{Аналоги}
    \textbf{Myscript}
        \begin{figure}
            \centering
            \includegraphics[width=1\linewidth]{Math}
            \label{fig:math}
        \end{figure}
\end{frame}
%----------------------------------------------------------------------------------------------------------
\begin{frame}{Аналоги}
    \begin{figure}
        \centering
        \includegraphics[width=1\linewidth]{Detexify}
        \label{fig:math}
    \end{figure}
\end{frame}
%----------------------------------------------------------------------------------------------------------
\begin{frame}{Аналоги}
    \begin{figure}
        \centering
        \includegraphics[width=1\linewidth]{ABBY}
        \label{fig:math}
    \end{figure}
\end{frame}
%----------------------------------------------------------------------------------------------------------
\begin{frame}{Датасет}
    \begin{figure}
        \centering
        \includegraphics[width=0.9\linewidth]{Dataset}
        \label{fig:math}
    \end{figure}
\end{frame}
%----------------------------------------------------------------------------------------------------------
\begin{frame}{Особенности нашей реализации}
    \begin{itemize}
        \item Работа в браузере
        \item Анализ всего документа
        \item Интеллектуальный онлайн-редактор
    \end{itemize}
\end{frame}
%----------------------------------------------------------------------------------------------------------
\begin{frame}{Дедлайны}
         \begin{enumerate}
            \item[1] Каждый погружается в свою задачу, изучение необходимых библиотек и фреймворков
             \item[2-3]  Примитивный сайт и latex редактор, возможность залогиниться и отправить картинку, предобработка изображения, распознавание русского текста
             \item[4] Наработки интеллектуального редактирования: подсвечивание сомнительных слов, исправление опечаток. Доведение до ума сайта.
             \item[5] разработка возможности распознавать формулы и прочие математические объекты
             \item[6] Распознавание каринок и графиков
             \item[7] Исправление багов и корректировка работы сервиса
        \end{enumerate}
\end{frame}

\end{document} 